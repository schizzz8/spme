\documentclass{article}

\begin{document}
	
	\section*{Semantic Perception, Mapping and Exploration}
	
	Acquisition and modeling of semantic information is a key requisite for mobile robots to be deployed in human environments. In this field, fundamental aspects faced by research are: the recognition of places and objects, the construction of semantic models and the exploration strategies to enrich contextual knowledge. In the remainder of this section, we will present relevant work that focused on the mentioned problems, namely: semantic perception, semantic mapping and semantic exploration.
	
	\subsection*{Semantic Perception}
	
	Mobile and autonomous robots operate in unstructured and unpredictable environments, where it's unlikely to have a prior knowledge of the workspace. In such settings, different techniques have been proposed to provide the robot with	the capability of extracting relevant information from sensor data to complete its tasks. In the last two decades, thanks to the increasing availability of visual and range sensors, a common solution has been to rely on scene analysis and image understanding methods.
	\subsection*{Semantic Mapping}
	
	\subsection*{Semantic Exploration}
	
	
\end{document}